This work was relevant due to the fact that all previous cyber security work in regards to low rate DDoS attacks could be seen as deeply flawed. For example, Tripathi generated their own data to validated their own hypothesis and methods. One of my goals was to show that cyber security work needs to be validated by real data. The work is also relevant due to an increasing reliance on websites and the internet in general. If there aren't effective ways to check attacks, then they have the potential to disrupt daily life. I saw this actualise itself when, after helping a Newcastle University Researcher I was able to reduce the running cost of their web severs by fifty percent. This shows the real world effect that the software has, along with the harm that attacks can cause if they are able to go undetected. The work was also relevant to me on a personal level; as it allowed me to create some research that has never been done before and go on to develop it as a product to sell to other websites.