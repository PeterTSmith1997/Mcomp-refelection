%!TeX root=main.tex

%The work carried out as part of this MCOMP has been challenging, largely due to factors benod my control
%Serious reflection is needed down below. 
Throughout this project, there have been many stages where reflection has been needed in order to understand the progress that has been made. By regularly putting my research on hold and taking a step back, I was able to develop a regular reflective habit  (\cite{dyment2010quality}). For example, after making changes to the formula, I would run the data again and reflect on the output. However, at other times throughout the year, there have been ongoing issues with a lack of available support in order to complete work. It added more stress and I often felt like I was rushing work that I should have taken more time over. There have been periods where reflection in action have been required, often when doing this we are unaware that we have derived meaning from an experience. Through this process we find ourselves doing the same action again, as it has a positive outcome (\cite{Schon83}). When I reflected on where I was in the project, and what I needed to do to progress it resulted in more effective decision making and problem solving. This was mainly evident by last minutes changes made to my proposal. This meant having to do an online zoom session, so that someone was able to assist me in completing my final draft. The positive outcome here was that I got the work done and I was able to learn that doing work on zoom, although not ideal, still contributed to the success of the project. One thing I'd change is, not asking other people to proofread my work near the deadline at risk of being too last minute to get any meaningful changes in.

The motivation behind this research was to solve a challenge that I faced, running my own website. The lack of literature in this area to do with cyber security was surprising, however this led to freedom in creating a new methodology. I had to combine some existing techniques and test some new ones. Due to the lack of relevant literature, the literature review was challenging to construct. Therefore it became more about identifying gaps in previous work. Hertz (1997) points out that 'the reflective researcher does not merely report the findings of the research but at the same time questions and explains how those findings are constructed'. (CITE) This shows that I was able to reflect on the available literature and ask questions that led to my hypothesis. A lot of these questions were not complicated in nature, but came from a working knowledge of websites and servers, the key lesson that I learned in this was that just by asking simple questions about the research, it was easy to identify flaws. Moving forward I will read research with a more critical point of view.



I felt highly motivated to do this work as it solves a real world issue that I had. Also, some people told me that doing this analysis as a mathematical model was impossible. It worked at undergrad level so I was motivated tp actually prove that, but I was unable to previously.

At the start of year motivation was high, quickly fell away when i found out that I had no support.

The fight with the uni to cone back next year increased my motivation, because I wanted to show hem what I could do.

After xmas break my motivation increased further as I actually got support more than one day a week.

Ethics 2nd rejection was a low point. "why bother if I can't get ethical approval?" and then Jamie ensured I got it.
