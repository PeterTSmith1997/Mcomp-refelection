%!TeX root=main.tex

%The work carried out as part of this MCOMP has been challenging, largely due to factors benod my control

Throughout this project, there have been many stages where reflection has been needed in order to understand the progress that has been made. At times, progress has felt limited and non-existent. However, these periods have allowed for reflection. By regularly putting my research on hold and taking a step back, I was able to develop a regular reflective habit  (\cite{dyment2010quality}). However, at other times throughout the year, there have been ongoing issues with a lack of available support in order to complete work. There have been periods where reflection in action have been required. Often when reflection in action, we are often unaware that we have derived meaning from an experience, however we find ourselves doing the same action again, as it has a positive outcome (\cite{Schon83}). By reflecting in action, it allows me to reflect on where I was in the project, and what I needed to do to progress thus resulting in more effective decision making and problem solving. This was mainly evident by last minutes changes made to my proposal, meaning I had to reflect on how to get this done. This meant having to do an online zoom session, so that someone was able to assess me to get changes done. The positive outcome here was that I got the work done and I was able to learn that doing work on zoom, although not ideal, still contributed to the success of the project.


The project undertaken was a continuation of my undergraduate project, looking at a new way to detect attacks on websites. The motivation of this was to solve a challenge that I faced, running my own website. The lack of literature in this area to do with cyber security was surprising, however this led to freedom in creating a new methodology. Due to the fact that there was no existing methodology, apart from my own, I had to combine some existing techniques and test some new ones. Due to the lack of relevant literature, the literature review was challenging to construct. Therefore it became more about identifying gaps in previous work. Hertz (1997) points out that 'the reflective researcher does not merely report the findings of the research but at the same time questions and explains how those findings are constructed'. (CITE) This shows that I was be able to reflect on the available literature and ask questions that led to my hypothesis. A lot of these questions were not complicated in nature, but came from a working knowledge of websites and servers, the key lesson that I learned in this was that just by asking simple questions about the research, it was easy to identify flaws 

While the ethical implications of this work will be discussed in a later section, it is important to reflect on the ethical process. Due to the ambiguity of user consent in the collection of website log data, and whether users had been informed about the collection and whether they had consented to the study. Websites say that data can be analysed however due to the university policy of expressed consent, there were questions about whether the data could be used. However, due to the study wanting real-world data, this might have changed user behaviour, knowing that they would've been tracked. This was done to mitigate the effects of possible confounding variables, such as social desirability. Studies have suggested that individuals online behaviour changed when they are being monitored.