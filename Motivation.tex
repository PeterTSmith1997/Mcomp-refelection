%!TeX root=main.tex

%The work carried out as part of this MCOMP has been challenging, largely due to factors benod my control
%Serious reflection is needed down below. 

The motivation behind this research was to solve a challenge that I faced, running my own website. The lack of literature in this area to do with cyber security was surprising, however this led to freedom in creating a new methodology. I had to combine some existing techniques and test some new ones. Due to the lack of relevant literature, the literature review was challenging to construct. Therefore it became more about identifying gaps in previous work. Hertz (1997) points out that 'the reflective researcher does not merely report the findings of the research but at the same time questions and explains how those findings are constructed'. (CITE) This shows that I was able to reflect on the available literature and ask questions that led to my hypothesis. A lot of these questions were not complicated in nature, but came from a working knowledge of websites and servers, the key lesson that I learned in this was that just by asking simple questions about the research, it was easy to identify flaws. Moving forward I will read research with a more critical point of view.

Throughout the project, my motivation fluctuated at differing points. As a student with cerebral palsy, I have an intrinsic motivation to succeed. My disability gives  me an obligation to prove myself in order to give myself a sense of purpose. This is parallel to \cite{bye2007motivation} who suggests that disabled students exhibit greater intrinsic motivation in comparison to non-disabled counterparts. At the start of the year, my motivation was high, due to the fact that I wanted to prove my methodology worked due to some people telling me that a mathematical model would be impossible. However, this quickly vanished due to a discourse with the university regarding my level of care. Literature indicates that disabled students that require care can achieve more when this care need is met. Even though my care needs were not met, I was adamant that I would not be taking a year out, as some had suggested and instead used this adversity as a source of motivation to produce a high-quality research paper. After Christmas, I secured an appropriate level of support from the university, as they agreed that I could reduce my timetable as well as resit some modules next year. This further increased my motivation as I now found that the project was achievable. Even when my motivation was low, I found that, by setting small achievable targets, and completing them would keep me motivated.