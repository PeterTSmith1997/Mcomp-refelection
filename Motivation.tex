%!TeX root=main.tex

%The work carried out as part of this MCOMP has been challenging, largely due to factors benod my control

Throughout this project, there have been many stages where reflection has been needed in order to understand the progress that has been made. At times, progress has felt limited and non-existent. However, these periods have allowed for reflection. By regularly putting my research on hold and taking a step back, I was able to develop a regular reflective habit \cite{dyment2010quality},However, at other times throughout the year, there have been ongoing issues with a lack of available support in order to complete work. There have been periods where reflection in action have been required (\cite{Schon83}).Often when reflection in action, we are often unaware that we have derived meaning from an experience, however we find ourselves doing the same action again, as it has a positive outcome. By reflecting in action, it allows me to reflect on where I was in the project, and what I needed to do to progress thus resulting in more effective decision making and problem solving. This was mainly evident by last minutes changes made to my proposal, meaning I had to reflect on how to get this done. This meant having to do an online zoom session, so that someone was able to assess me to get changes done. The positive outcome here was that I got the work done and I was able to learn that doing work on zoom, although not ideal, still contributed to the success of the project.


The project undertaken was a continuation of my undergraduate project, looking at a new way to detect attacks on websites. The motivation of this was to solve a challenge that I faced, running my own website. The lack of literature in this area to do with cyber security was surprising, however this led to freedom in creating a nbew methodology. Due to the fact that there was no existing methodology, apart from my own, I had to combine some exisitng techniques and test some new ones.

Hertz (1997, viii) notes that the reflective researcher does not merely report the findings of the research but at the same time questions and explains how those findings are constructed (LITERATIURE REVIEW)
