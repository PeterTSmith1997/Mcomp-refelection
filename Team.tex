The work for this module was meant to be undertaken as a group research project, however it was decided early on that I would complete it as an individual. Primarily due to the fact that the work was heavily linked to the work carried out for my undergrad dissertation. Most of the underlying software was written as a part of my dissertation, therefore most of the work was identifying small improvements that could be made. Then developing a methodology to prove that accuracy of the program. Whilst it may have been useful to get new ideas in, which may have led to a better outcome. If I were to have worked in a group the collaboration would have been beneficial as Robert E. Levasseur (2010) states, "to initiate and... the high level of two-way communication" (\cite{levasseur2010people})

Although I was primarily working alone, I did receive a lot of assistance from academic staff. This worked qite well due to the fact that the academic does not have any knowledge of cybersecurity, which meant that I was free to experiment and due to his knowledge of algorithms, he was able to advise all the methods used without getting caught up in the smaller details. This reslted in a more comphrensive approach to the problem; \citeauthor{thurow1999dynamics} argues that multidisciplinary settings allows individuals from different disciplines to contribute their disciplinary perspectives in an attempt to solve complex problems.

 However, this research has never been undertaken before; it may have been hard for new people to come in and understand the research.

When classifying the accuracy of the program it might have been beneficial to have someone to classify the data as well. to remove any bias that I might have had. 