The work for this module was meant to be undertaken as a group research project, however it was decided early on that I would complete it as an individual. Primarily due to the fact that the work was heavily linked to the work carried out for my undergrad dissertation. Most of the underlying software was written as a part of my dissertation, therefore most of the work was identifying small improvements that could be made. Then developing a methodology to prove that accuracy of the program. It may have been useful to get new ideas in, which may have led to a better outcome. If I were to have worked in a group the collaboration would have been beneficial as Robert E. Levasseur (2010) states, "to initiate and... the high level of two-way communication" (\cite{levasseur2010people}). Collaboration leads to greater success in group projects if done well. Despite not being in a group I was still able to have effective communication with a lecturer in the department. Although they did not have any knowledge of cyber-security, they did know a lot about programming algorithms. I was able to take ideas to them and they would know how to implement them, without disagreement over the direction the project took.

Because there was a lack of research in the area I was researching, it may have made it challenging for someone to join the project and be able to completely understand the complexity of the project. Theoretically by working alone it made the ethics application easier, as I relied on getting data from customer websites. They already knew who I was and as a result were more willing to share their data with me for this project as well. However, there were times when having more than one dedicated researcher would have been beneficial. For example, when classifying the accuracy of the program it might have been beneficial to have someone to classify the data as well. This would help to remove any bias that I might have had.

If this had been an entirely new project it would have been beneficial to collaborate with more than one person. To bring in new ideas and a different point of view. However due to the fact that it was a continuation I believe it was the right decision to do it on my own.  