Primarily I worked alone, although I did receive a lot of assistance from Nick (a Computer Science lecturer). \cite{thurow1999dynamics} argues that multidisciplinary settings allow individuals from different disciplines to contribute their disciplinary perspectives in an attempt to solve complex problems. Myself and Nick have complimentary skill-sets that aided in a multidisciplinary approach. We focussed on different areas of the research, with Nick able to use his knowledge of algorithms and programming to help improve the formula. Whilst I focussed on which aspects of the data that pertained to cyber security. Hence I was able to identify risk factors and Nick would then advise on how to efficiently calculate the risk attached to the data. For example, I determined that user agents could be an identifier of risk, I had originally tried a Levenshtein distance in order to find the nearest user agent (this did not come to fruition). Upon reflection of this approach, it may have resulted in misidentification of user agents due to their similarities. However, Nick suggested doing a bag of words approach, which was ultimately a stronger and more accurate methodology. This stopped me overcomplicating the detection of user agents. In future projects I will now begin by looking at simpler approaches first. Furthermore, Nick had prior knowledge of the project, due to supervision at undergraduate level and therefore understood the project and its complexities. 

Looking back on this project, I enjoyed working with Nick, due to our different approaches. Nick always had a positive outlook on the project and its challenges; this therefore kept my motivation high and I always came away from our meetings believing I would complete the project with ease. He also helped by, not only generating ideas for the project but helping to implement them. This further helped reduce any stress on me and he enabled me to have faith in our methodology. However, a slight problem was, Nick was difficult to reach at times. The primary way to ask a question to him was looking for a space in his calendar that was mutually convenient. I often felt bad for booking in meetings for issues that could have been solved via email but, this was unavoidable. This added to my stress, particularly at key times in the project; for example, when submitting my ethics for the third time. I was going to wait for Nick to double-check, but I was getting more stressed about the ethics not coming back on time; rather than the potential of it being rejected for the fourth time for problems Nick may have spotted. Overall, working with Nick proved beneficial and if I were to do a similar project I would ensure we had better communication outside of meetings if possible.

%The project was a sole effort, there was no instated leader, however, project management strategies were adopted in order to monitor and assure the completion of this quality project (\cite{visser1990expert}). Through building a framework with the option for additional modules to be added, I learned that I have competence as a project leader in terms of project management. The time management of the project was well overseen and time critical elements were put in place, however, the use of a GANTT chart was considered unfitting for the project. AJ Shenhar states that, \textit{"project management differs with the kind of project, and that management style, attitude, and practice must be adapted to the specific project type"} (\cite{Shenhar1}; 33). In short this means that different projects require different management styles. Using this project as an example, not having an initial GANTT chart led to a more agile approach.

%In a team related project there may have been issues regarding the overall build of the software including the login screen. An article by \citeauthor{GanttPRO} discusses the 10 major pitfalls in team based project management, one of the pitfalls being; setting unrealistic deadlines. (\cite{GanttPRO}). This may have lead to overall communication problems and further compatibility issues for adding further module content, due to deadlines not being met and the delay of the project. A great deal of skill development around project management would have to be achieved, especially with time management and communication skills.

%Keeping us both informed as to where the research was and what needed to be undertaken for the following week
Throughout the year myself and Nick critically evaluated our techniques. We developed a habit of meeting regularly, therefore maintaining momentum in the research overall. Despite this, there were still weeks where progress was slow. These made me feel like I hadn't done anything, negatively impacting my motivation. The project was not initially planned out, which naturally generated a more agile approach. Furthermore, due to the novel nature of the work there was a danger that I would set unrealistic deadlines in order to complete the research; which is deemed by \citeauthor{GanttPRO} to be one of ten major pitfalls of projects success(\cite{GanttPRO}). This approach yielded results because I had to deal with additional complication. The level of support I require was not met for an elongated period of time, which made it difficult to plan what activities could be undertaken in advance. Once, I received a phone call at 6pm, informing me I could come into University the next day. This is an example showing I had varying amounts of time to work on the project each week. In addition to this, I also had times were I thought I would be in and would find out that I could not attend. This meant some of our meetings had to take place on Teams and sometimes involved brainstorming on whiteboards. This was sometimes difficult to read, if internet connection was poor. On reflection the whiteboard feature on Teams may have been better; also, I should have tried to get more support online, if not in person. The circumstances dictated this work required a different approach to manage the project. This is inline with AJ Shenhar's theory on project management (\cite{Shenhar1}), who suggests that different projects require different styles of management. So due to external factors an agile project management style was needed.

%This occurred organically in the present research as ideas could be effectively implemented without disagreement, regarding the direction of the project.

%needs further reflection

The work was heavily linked with that of my undergraduate dissertation. Most of the underlying software was already written, therefore I worked to identify small improvements that could be made. Then developing a methodology to prove the accuracy of the program. However, new ideas from different researchers may have led to a better outcome, because of the opportunity for collaboration. It can be shown that the more collaboration that exists in research projects, the higher the quality of the research (\cite{figg2006scientific}). Therefore the collaboration within this project led to a better output but other collaborators might have increased this further. Myself and Nick would often challenge one another's ideas in order to ensure our methodology was robust. It may have been useful to include another cyber security researcher, in order to further challenge our thinking. Additionally, when classifying the accuracy of the program; having someone to classify the data as well, helping to remove any bias that I might have had. However, I did get a high degree of collaboration with Nick; this is in line with Robert E. Levasseur (2010) who states "to initiate and... the high level of two-way communication" (\cite{levasseur2010people}). The initiation process occurred very naturally between myself and Nick. While on placement, we discussed a problem caused by an IP address attacking my website and I proposed that I program something to detect the attack. This sparked a reciprocal flow of ideas that continued throughout the research, even if some elements were absent from the final project. 



If I were to do this again with the knowledge of my lack of support and doing a new project I would have worked in a larger group because the level of stress I put on myself would have been reduced. However, there may have been deadlines that I might have been unable to meet; potentially causing a greater amount of stress




I also found that when working in novel areas some of the concepts maybe difficult to explain, therefore continuity of the researchers no matter the size is important. If new people had joined at the start of this year it may have been difficult to explain the principles behind the methodology. 

I had already worked with Nick previously in undergrad, so my confidence communicating was already high. Throughout the year I gained more confidence communicating my ideas to various other people. For example, with my different support workers being able to communicate where I was up to in the project quickly and explain what they could do to help me. Furthermore, I had to communicate concisely with Jamie in order to get my ethics. 


%The initiation for this project was started while I was on placement. I chatted to Nick about a problem that I was having with an IP address attacking my website and made a comment about 'I should be able to programme something to identify the attack'. Nick then started drawing some user interfaces that didn't end up coming into the final project. However this initiation felt very natural and began some good two way communication with ideas flowing backwards and forwards, and set us up for good communication backwards and forewords due to the fact we both felt invested in the project and when communicating, we allowed each other space to come up with new ideas.
% However, this research has never been undertaken before; it may have been hard for new people to come in and understand the research.


%If this had been an entirely new project it would have been beneficial to collaborate with more than one person. To bring in new ideas and a different point of view. However due to the fact that it was a continuation I believe it was the right decision to do it on my own.  