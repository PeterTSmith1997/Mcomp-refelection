Although I was primarily working alone, I did receive a lot of assistance from academic staff. This worked quite well due to the fact that the academic does not have any knowledge of cybersecurity, which meant that I was free to experiment and he was able to advise all the methods used without getting caught up in the smaller details, due to his knowledge of algorithms. Furthermore, the academic staff had prior knowledge of the project, due to supervision at undergraduate level and therefore understood the project and its complexities. This resulted in a more comprehensive approach to the problem; \cite{thurow1999dynamics} argues that multidisciplinary settings allows individuals from different disciplines to contribute their disciplinary perspectives in an attempt to solve complex problems. This occurred organically in the present research as ideas could be effectively implemented without disagreement, regarding the direction of the project. An example of this is, when analysing the user agents, I had originally tried to implement a Levenshtein distance to try and find the nearest user agent. However, the academic suggested doing a bag of words approach, which was ultimately a stronger and more accurate methodology. 

Throughout the year myself and my supervisor tried to get into a regular habit of meeting. It also kept us both informed as to where the research was at and what needed to be undertaken for the following week. This also prevented delays in the research if I got stuck on something. This worked well to keep the momentum going and I was able to establish if a method was working or not. One of the downsides of this was if there had been a slow progress week it felt like I hadn't done anything, which negatively impacted my motivation. Also, some of our meetings involved brainstorming on whiteboards, this was sometimes difficult to read if the meeting was on Teams, due to bad internet connection. This would have been avoidable if my support had been properly arranged in advance of my return to University.

As there is a lack of research in the area I was researching, it may have made it challenging for someone to join the project and be able to completely understand the complexity of the project. Theoretically by working alone it made the ethics application easier, as I relied on getting data from customer websites. They already knew who I was and as a result were more willing to share their data with me for this project as well. However, there were times when having more than one dedicated researcher would have been beneficial. For example, when classifying the accuracy of the program it might have been beneficial to have someone to classify the data as well. This would help to remove any bias that I might have had.

The work for this module was meant to be undertaken as a group research project, however it was decided early on that I would complete it as an individual. Primarily due to the fact that the work was heavily linked to the work carried out for my undergrad dissertation. Most of the underlying software was written as a part of my dissertation, therefore most of the work was identifying small improvements that could be made. Then developing a methodology to prove that accuracy of the program. It may have been useful to get new ideas in, which may have led to a better outcome. If I were to have worked in a group the collaboration would have been beneficial as Robert E. Levasseur (2010) states, "to initiate and... the high level of two-way communication" (\cite{levasseur2010people}). Retrospectively, the initiation process occurred very naturally between myself and Nick. While on placement, we chatted about a problem that I was having with an IP address attacking my website and suggested that I programme something to detect the attack. This commenced a reciprocal flow of ideas that continued throughout the research, with some elements being absent from the final project. Despite not being in a group I was still able to have effective communication with a lecturer in the department. Although they did not have any knowledge of cyber-security, they did know a lot about programming algorithms. I was able to take ideas to them and they would know how to implement them, without disagreement over the direction the project took. It can be shown that the more collaboration that exists in research projects, the higher the quality of the research (\cite{figg2006scientific}). Therefore the collaboration within this project led to a better output but other collaborators would have increased this further. 


%The initiation for this project was started while I was on placement. I chatted to Nick about a problem that I was having with an IP address attacking my website and made a comment about 'I should be able to programme something to identify the attack'. Nick then started drawing some user interfaces that didn't end up coming into the final project. However this initiation felt very natural and began some good two way communication with ideas flowing backwards and forwards, and set us up for good communication backwards and forewords due to the fact we both felt invested in the project and when communicating, we allowed each other space to come up with new ideas.
% However, this research has never been undertaken before; it may have been hard for new people to come in and understand the research.


If this had been an entirely new project it would have been beneficial to collaborate with more than one person. To bring in new ideas and a different point of view. However due to the fact that it was a continuation I believe it was the right decision to do it on my own.  