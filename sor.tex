\subsection{Scope}

My research attempted to assess if website log files contained sufficient data for a formulaic approach to detecting attacks. Part of the scope was to collect a representative sample of real-world data, to measure the effectiveness of the proposed methodology. The use of this sample challenged conventional thinking that studies have to generate their own data.

The narrow scope of this work was correct due to its novel nature. This was evidenced by the difficulty I had obtaining ethical approval, which will be explored in a later section. Even given these challenges, I was able to gather enough data to test my hypothesis. The methodology was tested on one website but has the potential to be applied to any website. The variables associated with increased risk, such as country of origin, were controlled for. 

%However, due to countries' differing internet patterns, this technique may be hard to generalise to a global scale, especially for a pilot study.


%Mention research has not been attempted by others in the field.

\subsection{Objectives}

The objectives in my proposal were to; identify how attacks are evading current techniques, understand attack characteristics and refine a formula that can detect and determine risk. The high accuracy of the formula indicates I have understood current attacks and their evasion techniques proving a successful detection methodology.

Although, I removed an objective, which was to develop a more refined way to assess a country's risk. I intended to look at the number of attacks coming from IP addresses per capita. However, this proved unachievable because, this data is difficult to determine. Also, I was unable to contact the lecturer who suggested this idea; because, they now work at another university. Therefore, I kept the original methodology used in the previous research, which showed the importance of assigning risk to an IP's country of origin.

\subsection{Risk}

The biggest risk I identified was that the data within logfiles is sensitive, because they contain IP addresses (that were used to identify their location). Therefore, website owners may be unwilling to give me their data for analysis. However, after engaging with them, they were happy to do so. I informed them that the research could potentially inform them of attacks on their website; if any happened to be found, I could advise them on how they could be mitigated. 

Another risk associated with this project was the lack of previous literature. I questioned; as this methodology seemed relatively straight-forward, why had nobody tried this before? Was it because it was unsuccessful? This risk was mitigated as, even if the conclusion did not support my hypothesis I would still be adding knowledge to this field. Therefore, this was a worthwhile risk and one that I would gladly take again.