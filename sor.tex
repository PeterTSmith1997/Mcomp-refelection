\subsection{Scope}

This research attempted to look at whether there was enough data in website log files for a formula to be applied, in order to detect attacks. Part of the scope was to collect a representative sample of real world data to measure the effectiveness; not to generate my own data. I think the scope was correct due to the novel nature of the work and the difficulty I had getting ethical approval, which will be explored in a later section. But I was still able to gather enough data to test my hypothesis. The methodology was tested on a local website and could be applied to any across the world. The risks associated with this region were known and acted as a control. However due to country's different internet patterns, it would be difficult to understand different traffic patterns. 

%Mention research has not been attempted by others in the field.

\subsection{Objectives}

The objectives in my proposal were to; identify how attacks are evading current techniques, understand attack characteristics and refine formula that can detect and determine risk. I had to remove an objective, which would have been to develop a better way to assign risk to a country. I intended to look at the number of attacks coming from IP addresses within a country compared to the size of the population. However, due to time constraints I encountered due to a lack of support from the University. Coupled with the fact this data is difficult to determine. Another reason for its removal was that I was unable to contact the lecturer who suggested this idea, because they now work at another university. Therefore, I kept the original methodology used in the previous research. This was just looking at the number of attacks from within a country. Given the fact that the formula has worked, shows that I have understood current attacks and their evasion techniques proving a successful detection methodology.

\subsection{Risk}

In my proposal I identified the risk that website owners may be unwilling to give me their data for analysis. However, after engaging with them, they were more than happy to take part. Primarily because, the research could potentially inform them of attacks on their website; if any happened to be found, I could advise them on how they could be mitigated. The biggest risk associated with this project was the lack of previous work. There was a question in my mind that as this methodology seemed so simple and obvious, why had nobody tried this before? Was it because it was unsuccessful? This risk was mitigated due to the fact that, even if the conclusion did not support my hypothesis I would still be adding knowledge to this field. Therefore, this was a worthwhile risk to take and one that I would gladly take again.