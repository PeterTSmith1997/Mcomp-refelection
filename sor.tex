\subsection{Scope}

The scope of this work was to continue the work started in my undergraduate project and to validate the methodology using real world data. It was part of the scope not to generate my own data and collect a representative sample of real world data to measure the effectiveness. The methodology was tested on a local website. The risks associated with this region were known and acted as a control. The methodology could be applied to any website across the world. However due to country's different internet patterns., it would be difficult to understand different traffic patterns. 



\subsection{Objectives}

I had to remove an objective, which would have been to develop a better way to assign risk to a country. I intended to look at the number of attacks coming from IP addresses within a country compared to the size of the population. However, due to time constraints I encountered due to a lack of support from the University, I kept the current methodology. This was just looking at the number of attacks from within a country. The other objectives in my proposal were to identify how attacks are evading current techniques, understand attack characteristics and refine formula and develop an updtaes formula that can detect and determine risk. I think the fact that the formula has worked shows that I have understood current attacks and their evasion techniques. I used this knowledge to refine the formula into a successfuly detection methodology.

\subsection{Risk}

The biggest risk associated with this project was the lack of previous work. Because, there was a question in my mind that this methodology seemed so simple and obvious, why had nobody tried this before? Was it because it did not work? This risk was mitigated due to the fact that, even if the conclusion was that this methodology did not work it would still be a valid conclusion to make.