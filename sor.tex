\subsection{Scope}

This research attempted to look at whether there was enough data in website log files for a formula to be applied, in order to detect attacks. Part of the scope was to collect a representative sample of real-world data to measure the effectiveness of the proposed methodology. By using real-world data I could prove it is more accurate when validating security models; therefore, challenging conventional thinking that studies have to generate their own data.

The scope of this work was correct due to the novel nature of the work and the difficulty I had getting ethical approval, which will be explored in a later section. Even given these challenges, I was still able to gather enough data to test my hypothesis. The methodology was tested on a local website and could be applied to any across the world. The risks associated with this region were known and acted as a control. However due to countries' different internet patterns, it would be difficult to understand different traffic patterns. 

%Mention research has not been attempted by others in the field.

\subsection{Objectives}

The objectives in my proposal were to; identify how attacks are evading current techniques, understand attack characteristics and refine a formula that can detect and determine risk. The high accuracy of the formula indicates I have understood current attacks and their evasion techniques proving a successful detection methodology. Although, I had to remove an objective, which would have been to develop a better way to assign risk to a country. I intended to look at the number of attacks coming from IP addresses within a country compared to the size of the population. But, I was unable to achieve this. I had issues with my support throughout most of the year, which forced time-constraints upon me; also, this data is difficult to determine. Another reason for its removal was that I was unable to contact the lecturer who suggested this idea, because they now work at another university. Therefore, I kept the original methodology used in the previous research, which still was able to prove that looking at risk for a country is a valid factor to consider.

\subsection{Risk}

In my proposal I identified the risk that website owners may be unwilling to give me their data for analysis. However, after engaging with them, they were happy for their data to be used. The research could potentially inform them of attacks on their website; if any happened to be found, I could advise them on how they could be mitigated. The biggest risk associated with this project was the lack of previous work. There was a question in my mind that as this methodology seemed relatively straight-forward, why had nobody tried this before? Was it because it was unsuccessful? This risk was mitigated due to the fact that, even if the conclusion did not support my hypothesis I would still be adding knowledge to this field. Therefore, this was a worthwhile risk and one that I would gladly take again.