\subsection{Scope}

The scope of this work was to continue the work started in my undergraduate project and to validate the methodology using real world data. It was part of the scope not to generate my own data and collect a representative sample of real world data to measure the effectiveness. The methodology was tested on a local website. The risks associated with this region were known and acted as a control. The methodology could be applied to any website across the world. However due to country's different internet patterns, it would be difficult to understand different traffic patterns. I think the scope was correct due to the novel nature of the work and the difficulty I had getting ethical approval, which will be explored in a later section. But I was still able to gather enough data to test my hypothesis.



\subsection{Objectives}

I had to remove an objective, which would have been to develop a better way to assign risk to a country. I intended to look at the number of attacks coming from IP addresses within a country compared to the size of the population. However, due to time constraints I encountered due to a lack of support from the University, Another reason for its removal was that I was unable to reach the lecturer who suggested this idea, because they now work at another university. Therefore, I kept the original methodology used in the previous research instead of trying the new methodology. This was just looking at the number of attacks from within a country. The other objectives in my proposal were to; identify how attacks are evading current techniques, understand attack characteristics and refine formula and develop an updated formula that can detect and determine risk. I think the fact that the formula has worked shows that I have understood current attacks and their evasion techniques. I used this knowledge to refine the formula into a successful detection methodology.

\subsection{Risk}

In my proposal I identified the risk that website owners may be unwilling to give me their data for analysis. However, after talking to website owners they were more than happy to give their data for analysis. Largely due to the fact they wanted to know if their website was under attack; if it was how the attack could be stopped. The biggest risk associated with this project was the lack of previous work. Because, there was a question in my mind that this methodology seemed so simple and obvious, why had nobody tried this before? Was it because it did not work? This risk was mitigated due to the fact that, even if the conclusion was that this methodology did not work it would still be a valid conclusion to make.