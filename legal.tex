\subsection{Ethics and Legal Issues}

Due to complications surrounding consent, ethical approval provided a significant challenge. This study worked using automatically collected data, as a result of this, informed consent could not have been obtained easily. Conventional practice stipulates the requirement for explicit consent before using personal data, resulting in repeated rejection. Retrospectively, a better approach would have been to reference the lawful and relevant use of GDPR in data collection. Having an impartial reviewer giving input on the form may have been beneficial as they could've identified any issues before submission. Furthermore, after the first rejection, it would've been beneficial to enquire who could give appropriate guidance on the issue of consent, which may have led me to Jamie Mahoney quicker. 

Jamie was able to point me towards article 14 paragraph 5 of GDPR(\cite{european_commission_regulation_2016}) which states that informed consent inst necessary under the conditions that the provision of such information proves impossible or would involve a disproportionate effort. By citing this information, I was then able to receive ethical approval. After discussing my ethics with other researchers, there was also a suggestion that I could use data that was greater than 6 months old as this falls out of the purview of data protection. Furthermore, I could've tried to anonymise the data in order to simplify the ethics, however this may have added complications to the project. For example, I would be unable to determines the location of an IP address.