\subsection{Ethics}
Due to the ambiguity of user consent in the collection of website log data, and whether users had given informed consent about the collection. Websites say that data can be analysed however due to the university policy of expressed consent, there were questions about whether the data could be used. However, due to the study wanting real-world data, this might have changed user behaviour, knowing that they would've been tracked. This was done to mitigate the effects of possible confounding variables, such as social desirability. Studies have suggested that individuals online behaviour changed when they are being monitored.

Another ethical implication of the work may be the fact that  entire country is given a risk. This is mainly done to give the software an idea of context, however a user from a high risk country can still get a low overall risk score.

\section{Legal Issues}
There was a potential legal issue as people could not opt out of the data analysis. This is due to the fact that as soon as they went on to the website their IP address was logged. Most websites privacy policy state that IP addresses will be logged and used for analysis, therefore most users should be aware of how their data will be used.