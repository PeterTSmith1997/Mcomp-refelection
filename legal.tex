\subsection{Ethics and Legal Issues}
There was a delay in receiving ethical approval due to the ambiguity of user consent in the collection of website log data, and whether users had given informed consent about the collection. Websites say that data can be analysed however due to the university policy of expressed consent, there were questions about whether the data could be used. However, due to the study wanting real-world data, this might have changed user behaviour, knowing that they would've been tracked. After my ethics form asked for clarification the second time, I met with Jamie Mahoney who advised that we should look at GDPR. In particular article 14 paragraph 5, which states if it is too much effort to get a form of consent then you do not need it (look at the actual statement). Upon reflection I should have contacted Jamie earlier or completed further research alone of GDPR. This is also a legal issue and it is the same as the ethical problem.

Ideally I would have liked to have had the ethics in sooner in the year, but it could only come after the proposal was marked. I couldn't rally do anything with that. Should have possibly asked someone on the module to peer review my ethics form. Could I have generated my own data? 

COuld have used a website's old data, providing it was over 6 months old; after this point it falls out of the purview of data protection

Another ethical implication of the work may be the fact that  entire country is given a risk. This is mainly done to give the software an idea of context, however a user from a high risk country can still get a low overall risk score.

Should have anonymised the IP addresses as much as possible by using a hash function, however that may have made any analysis more difficult. I had known how complex the ethixs wpld be for this project and given the time-constraints. I maybe should have done a simpilar project.

There was a potential legal issue as people could not opt out of the data analysis. This is due to the fact that as soon as they went on to the website their IP address was logged. Most websites privacy policy state that IP addresses will be logged and used for analysis, therefore most users should be aware of how their data will be used.